\chapter{Fazit und Ausblick} 
\label{ch:Fazit und Ausblick}

Die Bachelorarbeit hatte das Ziel eine intelligente Ladestation in eine Multicopter Simulation zu integrieren.
Daf�r wurden in Kapitel \ref{cha:Anforderungsanalyse} Anforderungen aufgenommen und kategorisiert. Wie in der Auswertung der Anforderungen \ref{sec:Auswertung Anforderungen} festgestellt wurde, konnten die wichtigsten Anforderungen erf�llt werden.
Die Ladestation verf�gt �ber eine definierbare, variable Anzahl an Warte- und Ladepl�tzen. Sie kann mit den anderen Netzwerkteilnehmern, in erster Linie Multicopter, kommunizieren und Informationen �ber den Ladeverlauf vorhersagen. Dar�ber hinaus k�nnen Reservierungen der Ladepl�tze vorgenommen werden, die einen vorhersehbaren Simulationsverlauf erm�glichen. Der Ladeverlauf wird realit�tsnah abgebildet, dennoch sind wie in Kapitel \ref{sec:Fazit} festgestellt die Beispielmessungen nicht repr�sentativ. Durch die �hnlichkeit mit den in der Literatur festgestellten Angaben, k�nnen die Werte ungeachtet der M�ngel verwendet werden.
Die Ladestation ist dazu f�hig den nicht durchgehend linearen Ladeverlauf zu ber�cksichtigen und kann, wenn gew�nscht, den effizienteren Ladeanteil priorisieren. 
Alle mit der Ladestation in Kontakt stehenden Multicopter werden von dieser verwaltet und k�nnen auf Anfrage f�r neue Aufgaben zur Verf�gung gestellt werden. Dabei wird ein m�glichst wenig geladener und dennoch ausreichender Multicopter von der Ladestation ausgew�hlt, um m�glichst wenig Ressourcen zu binden.

F�r zuk�nftige Arbeiten bieten sich vor allem zwei sehr verschiedene Gebiete an.
Erstens k�nnten die Testmessungen stark erweitert werden. Daf�r w�re eine breitere Auswahl der Ladeger�te und Akkumulatoren mit unterschiedlichen Maximalkapazit�ten sinnvoll. Gest�tzt durch eine ausf�hrlichere Literaturrecherche spezialisiert auf den Themenbereich Ladeprozess f�r Lithium-Ionen-Akkumulatoren k�nnte dadurch der Ladealgorithmus verbessert werden und realit�tsn�here Ergebnisse liefern.
Zweitens beschr�nkt sich die Kommunikation der Ladestation nahezu ausschlie�lich auf Multicopter. Damit der globale Ladeprozess aller in der Simulation befindlicher Multicopter verbessert werden k�nnte, k�nnten die Ladestationen untereinander kommunizieren. 
Ans�tze daf�r bieten die nicht erf�llten funktionalen Anforderungen aus \ref{cha:Anforderungsanalyse}. Durch den Austausch von Multicoptern w�hrend und nach dem Ladeprozess k�nnte dieser beschleunigt werden und au�erdem der Bedarf von Ersatzmulticoptern besser bedient werden.
