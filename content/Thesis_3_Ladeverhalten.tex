\chapter{Ladeverhalten} 
\label{ch:Ladeverhalten}
Im folgenden Kapitel wird das Ladeverhalten der Ladestation entworfen. Begonnen wird mit dem Vergleich mehrerer Energiespeicher. Dieser Vergleich soll herausarbeiten, aus welchen Gr�nden in mobilen Anwendungen vorzugsweise Lithium-Ionen basierte Akkumulatoren eingesetzt werden. 
Darauf aufbauend werden �bliche Ladeverfahren und deren Spezifika untersucht. 
Nach dieser von Literatur angetriebenen Untersuchung, werden Beispielwerte eines Ladeger�tes und Akkumulators gemessen.
Abschlie�end wird ein Fazit gezogen, welches die Grundlage f�r den zu implementierenden Ladealgorithmus liefert.

\section{Energiespeicher}
\label{sec:Energiespeicher}
Im einleitenden Kapitel �ber Multicopter \todo{verweis}, wurde festgestellt, dass diese �blicherweise von Elektromotoren angetrieben werden. Elektromotoren ben�tigen eine elektrische Energiezufuhr, welche von Akkumulatoren bereit gestellt wird.

Akkumulatorentypen lassen sich in Gruppen basierend auf ihren chemischen Bestandtteilen einteilen. Drei h�ufig verwendete Typen basieren auf Lithium-Ionen, Nickel und Bleis�ure. Durch die gleiche Basis haben Akkumulatoren innerhalb einer Gruppe �hnliche Merkmale.
F�r die Auswahl eines Energielieferanten sind neben den funktionalen Eigenschaften Energiedichte (Wh/kg), Kapazit�t (Ah), Spannung pro Zelle (V) auch weitere nicht funktionale Eigenschaften wie Langlebigkeit, Sicherheit und Ladegeschwindigkeit relevant. In der Anwendung sind die wirtschaftlichen Aspekte ebenfalls ein relevanter Faktor. F�r die rein wissenschaftliche Betrachtung ergeben sich hier viele Probleme. Unterschiedliche Marktpreise basieren nicht zwingend auf der Wertigkeit der verwendeten G�ter, sondern k�nnen auch durch Skaleneffekte der Massenproduktion, Gesetze oder Wettbewerb bez�glich der Technologie entstehen. Diese Effekte wiederum basieren unter anderem auf der Nachfrage nach dem Produkt und damit auch auf der Bekanntheit. Dessen ungeachtet soll die Ladestation f�r aktuelle Quadrokopter konzeptioniert werden und nicht f�r eine Utopie.

\subsection{Bleis�ure Akkumulatoren}
\label{subsec:Bleis�ure Akkumulatoren}
In einem Bleis�ure Akkumulator besteht die Elektrode aus Blei und Bleioxid, der dazugeh�rige Elektrolyt besteht aus verd�nnter Schwefels�ure. Dieser Akkumulatortype gilt als der �lteste Entwurf und reicht bis in das 19. Jahrhundert zur�ck. 
Die Nennspannung einer Zelle betr�gt 2 V und ist damit ebenso wie die Energiedichte mit ungef�hr 30-50 Wh/kg vergleichsweise gering. Die Lebensdauer kann grob in einer Zeitspanne angegeben werden, w�hrend die anderen Gruppen vor allem durch die Nutzung abgebaut werden. Je nach Qualit�t ist eine Lebensdauer von 4-15 Jahren zu erwarten. \todo{wirklich?}
Durch die lange Bekanntheit der Technologie, sind diese Akkumulatoren relativ g�nstig.

Einsatzgebiete sind aufgrund des hohen Gewichts vor allem station�re Anwendungen, beispielsweise Notstromversorgung und Energiespeicher f�r Photovoltaikanlagen. In kleinerem Ausma� gibt es auch mobile Anwendungsgebiete, unter anderem als Starterbatterie in Autos werden h�ufig auf Bleis�ure basierende Akkumulatoren eingesetzt. \todo{Quelle}

\subsection{Nickel Akkumulatoren}
\label{subsec:Nickel Akkumulatoren}

\subsection{Lithium-Ionen Akkumulatoren}
\label{subsec:Lithium-Ionen Akkumulatoren}


\section{Ladeverfahren}
\label{sec:Ladeverfahren}

\section{Beispielmessungen}
\label{sec:Beispielmessungen}

\section{Fazit}
\label{sec:Fazit}
