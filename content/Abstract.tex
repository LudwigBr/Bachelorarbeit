\chapter*{Zusammenfassung}
\thispagestyle{empty}

Die vorliegende Bachelorarbeit integriert eine intelligente Ladestation in eine Simulationsumgebung f�r Multicopter.

Dabei wird ein realistisches Ladeverhalten simuliert. Der Ladealgorithmus basiert auf empirischen Daten. Die gemessenen Werte zeigen Gemeinsamkeiten mit den von der Literatur pr�sentierten Daten. Im �blicherweise eingesetzten Ladeverfahren f�r Lithium-Ionen-Akkumulatoren, Constant Current Constant Voltage (CCCV), wird ein gro�er Anteil der Gesamtkapazit�t linear aufgeladen. Die Ladegeschwindigkeit f�r den kleineren abschlie�enden Teil sinkt im Vergleich stark. 

Im Mittelpunkt der objektorientierten Konzeptionierung und Implementierung stehen die Aggregation, Aufbereitung und Kommunikation von Daten, die den Ladeprozess betreffen. Der Nachrichtenaustausch mit den Multicoptern dient der Verbesserung der Geschwindigkeit und Verl�sslichkeit des Ladeprozesses. Die Ladestation kann unter anderem eine Vielzahl von Multicoptern verwalten und deren Ladeprozess organisieren, den Ladeverlauf prognostizieren und einen passenden Ersatzmulticopter zur Verf�gung stellen, falls ein solcher in Verbindung mit der Ladestation steht. Dar�ber hinaus kann kurzfristig die geladene Gesamtkapazit�t erh�ht werden. Das wird durch die Priorisierung der Multicopter, die sich innerhalb der linearen Ladephase befinden, erreicht.