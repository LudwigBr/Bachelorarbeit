
\chapter{Einleitung}
\label{sec:haupt}

Ladeger�te sind allgegenw�rtig. Jeder Laptop und jedes Smartphone m�ssen regelm��ig aufgeladen werden. Solange jedes Ger�t sein eigenes Ladeger�t hat, scheint eine geordnete Verwaltung unn�tz. Doch sobald man sich vorstellt, dass das Smartphoneladeger�t mit zehn anderen Smartphonebesitzern geteilt werden soll, wird die Relevanz eines organisierten Ladeprozesses deutlich. Analog verh�lt es sich mit Ladestationen f�r Multicopter. Zus�tzlich ist das Verh�ltnis von Nutzungsdauer zu Ladezeit bedeutend kleiner, verglichen mit einem modernen Smartphone. Der Ladeprozess ist dadurch ein gr��erer und wichtigerer Bestandteil des Multicopters. Dar�ber hinaus wird nicht nur ein einzelnes Ladeger�t betrachtet, sondern unterschiedlich viele Ladestationen an unterschiedlichen geographischen Orten innerhalb der Simulation.
Einfache Ladeger�te k�nnen nicht kommunizieren. Dadurch k�nnen Informationen die zur Verbesserung des Ladeprozesses beitragen nicht geteilt werden. Die aktuelle gesch�tzte Wartezeit oder gar eine Prognose f�r eine vollst�ndige Aufladung k�nnten dem Multicopter bei der Auswahl der passenden Ladestation behilflich sein. 
Simulationsmodelle sind Abbilder der Realit�t. Sie werden erstellt um R�ckschl�sse auf die modellierte Realit�t zu erm�glichen. Der Ladealgorithmus tr�gt zur �bertragbarkeit der gewonnen Informationen bei und sollte deshalb m�glichst realistisch gestaltet sein.

Diese Bachelorarbeit hat das Ziel einen organisierten und intelligenten Ladeprozess in eine Simulation mit einer Vielzahl von Multicoptern und einer Vielzahl von Ladestationen zu integrieren. Kernproblem dabei ist die Aggregation, Aufbereitung und Weitergabe von Informationen die den Ladeprozess betreffen. Dadurch sollen Verbesserungspotentiale aufgedeckt werden, aus denen R�ckschl�sse f�r Realweltprojekte gezogen werden k�nnen.

