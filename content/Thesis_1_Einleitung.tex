
\chapter{Einleitung}
\label{sec:haupt}



\section{Motivation}
\label{sec:haupt:motivaton}

Energie ist eine allgegenw�rtige physikalische Gr��e. Elektrische Energie steht in den Industriestaaten durch einfachen Steckdosenzugriff in gro�er Menge zur Verf�gung. Der sparsame Umgang mit diesem Gut schont die finanziellen Ressourcen und 
reduziert die negativen Auswirkungen der Stromerzeugung auf unsere Umwelt. 


Die unbemannte Luftfahrt ist unter anderem f�r ihren moralisch bedenklichen Einsatz f�r milit�rische Zwecke bekannt. Davon abgesehen werden Drohnen auch in zivilen Bereichen immer verbreiteter. Gro�e Logistikunternehmen wie DHL oder Amazon nutzen Prototypen f�r die unbemannte Auslieferung von Paketen. Die Suche nach Opfern eines Lawinenungl�cks ist eine �u�erst gef�hrliche Aufgabe. Weitere Lawinen k�nnen ausgel�st werden oder andere Gefahren werden von der Schneedecke verborgen.

W�hrend die Nutzungsm�glichkeiten von unbemannten Luftfahrzeugen grundverschieden sind, wird Energie ben�tigt. 
Die Speicherung gr��erer Mengen von Energie ist ein anhaltendes Problem, neue Akkumulatoren verbessern die Situation, doch das Grundproblem bleibt bestehen. Energie ist knapp und sie zu transportieren zieht Aufwand, oft in Form von Gewicht, nach sich. Gerade f�r mobile Ger�te ist das Gewicht ein ausschlaggebender Punkt.
F�r jede, nicht in k�rzester Zeit l�sbare, Aufgabe muss die Stromversorgung der flugf�higen Helfer gesichert werden.  


\section{Problemstellung}
\label{sec:haupt:problemstellung}
Ladestationen sind ein wichtiger Bestandteil von Elektromobilit�t jeglicher Art.  


\section{Zielsetzung}
\label{sec:haupt:zielsetzung}

